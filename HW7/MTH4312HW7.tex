% !TEX root = MTH4312HW7.tex
\documentclass[12pt]{article}
\usepackage[margin=1in]{geometry} 
\usepackage{amsmath,amsthm,amssymb,scrextend}
\usepackage{fancyhdr}
\pagestyle{fancy}

\newcommand{\cont}{\subseteq}
\usepackage{tikz}
\usepackage{pgfplots}
\usepackage{amsmath}
\usepackage[mathscr]{euscript}
\let\euscr\mathscr \let\mathscr\relax% just so we can load this and rsfs
\usepackage[scr]{rsfso}
\usepackage{amsthm}
\usepackage{caption}
\usepackage{amssymb}
\usepackage{enumitem}
\usepackage{multicol}
\usepackage{etoolbox}
\usepackage{tcolorbox}
\usepackage{mdframed}
\usepackage{changepage}
\usepackage{units}
\usepackage[table]{xcolor}
\usepackage[colorlinks=true, pdfstartview=FitV, linkcolor=blue,
citecolor=blue, urlcolor=blue]{hyperref}

\DeclareMathOperator{\arcsec}{arcsec}
\DeclareMathOperator{\arccot}{arccot}
\DeclareMathOperator{\arccsc}{arccsc}
\newcommand{\ddx}{\frac{d}{dx}}
\newcommand{\dfdx}{\frac{df}{dx}}
\newcommand{\ddxp}[1]{\frac{d}{dx}\left( #1 \right)}
\newcommand{\dydx}{\frac{dy}{dx}}
\let\ds\displaystyle
\newcommand{\intx}[1]{\int #1 \, dx}
\newcommand{\intt}[1]{\int #1 \, dt}
\newcommand{\defint}[3]{\int_{#1}^{#2} #3 \, dx}
\newcommand{\imp}{\Rightarrow}
\newcommand{\un}{\cup}
\newcommand{\inter}{\cap}
\newcommand{\ps}{\mathscr{P}}
\newcommand{\set}[1]{\left\{ #1 \right\}}
\newenvironment{answer}
    {\begin{adjustwidth}{0pt}{0pt}}
    {\end{adjustwidth}}

\newtheorem*{sol}{Solution}
\newtheorem*{claim}{Claim}
\newtheorem{problem}{Problem}
\theoremstyle{remark}  % Style for remarks
\newtheorem*{remark}{Remark}

\begin{document}
\setlength{\abovecaptionskip}{0pt} % Reduce space above caption
% EVERYTHING ABOVE THIS LINE IS JUST PREABLE, NO NEED TO MESS WITH IT.__________________________________________________________________________________________

\lhead{Rawson Duplantis}
\chead{MTH 4312: Cryptology}
\rhead{\today}

% 2.1.1: Exercises 2 & 4; 2.1.3: Exercises 1, 3, & 4
\stepcounter{section}
\section{Quantum Mechanics}
\subsection{Photon Polarization}
\subsubsection{Linear Polarization}
\stepcounter{problem}
%\begin{problemgroup}
    \begin{problem}
        Consider the matrix $R=\begin{pmatrix}
            \cos\phi & -\sin\phi \\
            \sin\phi & \cos\phi
        \end{pmatrix}$.
        \begin{enumerate}[label=(\alph*)]
            \item Show that $R$ is an orthogonal matrix.
            \item Apply $R$ to the linear polarization state $|s\rangle = \begin{pmatrix}
                \cos\theta \\
                \sin\theta
            \end{pmatrix}$. Describe in everyday language the effect of this transformation on a state of linear polarization.
        \end{enumerate}
    \end{problem}
    \begin{answer}
        Answer here...
    \end{answer}
%\end{problemgroup}
\stepcounter{problem}
%\begin{problemgroup}
    \begin{problem}
        Let $R=\begin{pmatrix}
            \cos(\pi/(2n)) & -\sin(\pi/(2n)) \\
            \sin(\pi/(2n)) & \cos(\pi/(2n))
        \end{pmatrix}$.
        \begin{enumerate}[label=(\alph*)]
            \item Compute $R^n$. That is, compute the product of $n$ factors of $R$, where the multiplication is matrix multiplication. You may find the following trigonometric identities helpful: $\cos\alpha\cos\beta - \sin\alpha\sin\beta = \cos(\alpha + \beta)$; $\cos\alpha\sin\beta + \sin\alpha\cos\beta = \sin(\alpha + \beta)$.
            \item A horizontally polarized photon passes successively through $n$ small containers of sugar water, each of which effects the transformation $R$. The photon then encounters a polarizing filter whose preferred axis is horizontal. What is the probability of the photon passing the filter?
            \item Another horizontally polarized photon passes through the same $n$ containers of sugar water. But now, just after each container there is a polarizing filter whose preferred axis is horizontal. What is the probability that the photon will pass through all $n$ filters?
            \item Find the limit of your answer to part (c) as $n$ approaches infinity.
        \end{enumerate}
    \end{problem}
    \begin{answer}
        Answer here...
    \end{answer}
%\end{problemgroup}
\setcounter{subsubsection}{2}
\subsubsection{Circular and Elliptical Polarization}
\setcounter{problem}{0}
%\begin{problemgroup}
    \begin{problem}
        For each of the following state vectors, find a normalized vector that is orthogonal to the given vector. $$\begin{pmatrix}
            \nicefrac{\sqrt{3}}{2} \\
            \nicefrac{1}{2}
        \end{pmatrix}\quad\begin{pmatrix}
            \nicefrac{1}{\sqrt{2}} \\
            \nicefrac{i}{\sqrt{2}}
        \end{pmatrix}\quad\begin{pmatrix}
            \nicefrac{1-i}{2} \\
            \nicefrac{1+i}{2}
        \end{pmatrix}$$
    \end{problem}
    \begin{answer}
        Answer here...
    \end{answer}
%\end{problemgroup}
\stepcounter{problem}
%\begin{problemgroup}
    \begin{problem}
        The rotation operation $R_\phi$, defined by $$R_\phi=\begin{pmatrix}
            \cos\phi & -\sin\phi \\
            \sin\phi & \cos\phi
        \end{pmatrix}$$ rotates any linear polarization state by an angle $\phi$. What does this transformation do to the right-hand circular polarization state? Is the resulting state a state of linear polarization, circular polarization, or elliptical polarization? Does the answer to this question depend on the value of $\phi$?
    \end{problem}
    \begin{answer}
        Answer here...
    \end{answer}
%\end{problemgroup}
%\begin{problemgroup}
\begin{problem}
    Consider the elliptical polarization represented by $|s\rangle = \begin{pmatrix}
        \nicefrac{1}{\sqrt{2}} \\
        \nicefrac{1+i}{2}
    \end{pmatrix}$. Suppose the measurement $M_\theta$ of Example 2.1.6 is applied to a photon in the state $|s\rangle$.
    \begin{enumerate}[label=(\alph*)]
        \item Find the probabilities of the two outcomes as functions of $\theta$.
        \item For what value of $\theta$ do the two probabilities differ the most from each other? The basis defined by $M_\theta$ for this value of $\theta$ can be thought of as giving the "principal axes" of the elliptical polarization.
    \end{enumerate}
\end{problem}
\begin{answer}
    Answer here...
\end{answer}
%\end{problemgroup}

\end{document}