% !TEX root = MTH4312HW7.tex
\documentclass[12pt]{article}
\usepackage[margin=1in]{geometry} 
\usepackage{amsmath,amsthm,amssymb,scrextend}
\usepackage{fancyhdr}
\pagestyle{fancy}

\newcommand{\cont}{\subseteq}
\usepackage{tikz}
\usepackage{pgfplots}
\usepackage{amsmath}
\usepackage[mathscr]{euscript}
\let\euscr\mathscr \let\mathscr\relax% just so we can load this and rsfs
\usepackage[scr]{rsfso}
\usepackage{amsthm}
\usepackage{caption}
\usepackage{amssymb}
\usepackage{enumitem}
\usepackage{multicol}
\usepackage{etoolbox}
\usepackage{tcolorbox}
\usepackage{mdframed}
\usepackage{changepage}
\usepackage{units}
\usepackage[table]{xcolor}
\usepackage[colorlinks=true, pdfstartview=FitV, linkcolor=blue,
citecolor=blue, urlcolor=blue]{hyperref}

\DeclareMathOperator{\arcsec}{arcsec}
\DeclareMathOperator{\arccot}{arccot}
\DeclareMathOperator{\arccsc}{arccsc}
\newcommand{\ddx}{\frac{d}{dx}}
\newcommand{\dfdx}{\frac{df}{dx}}
\newcommand{\ddxp}[1]{\frac{d}{dx}\left( #1 \right)}
\newcommand{\dydx}{\frac{dy}{dx}}
\let\ds\displaystyle
\newcommand{\intx}[1]{\int #1 \, dx}
\newcommand{\intt}[1]{\int #1 \, dt}
\newcommand{\defint}[3]{\int_{#1}^{#2} #3 \, dx}
\newcommand{\imp}{\Rightarrow}
\newcommand{\un}{\cup}
\newcommand{\inter}{\cap}
\newcommand{\ps}{\mathscr{P}}
\newcommand{\set}[1]{\left\{ #1 \right\}}
\newenvironment{answer}
    {\begin{adjustwidth}{0pt}{0pt}}
    {\end{adjustwidth}}

\newtheorem*{sol}{Solution}
\newtheorem*{claim}{Claim}
\newtheorem{problem}{Problem}
\theoremstyle{remark}  % Style for remarks
\newtheorem*{remark}{Remark}
\renewcommand*{\thefootnote}{\fnsymbol{footnote}}

\begin{document}
\setlength{\abovecaptionskip}{0pt} % Reduce space above caption
% EVERYTHING ABOVE THIS LINE IS JUST PREABLE, NO NEED TO MESS WITH IT.__________________________________________________________________________________________

\lhead{Rawson Duplantis}
\chead{MTH 4312: Cryptology}
\rhead{\today}

% 2.1.1: Exercises 2 & 4; 2.1.3: Exercises 1, 3, & 4
\stepcounter{section}
\section{Quantum Mechanics}
\subsection{Photon Polarization}
\subsubsection{Linear Polarization}
\stepcounter{problem}
%\begin{problemgroup}
    \begin{problem}
        Consider the matrix $R=\begin{pmatrix}
            \cos\phi & -\sin\phi \\
            \sin\phi & \cos\phi
        \end{pmatrix}$.
        \begin{enumerate}[label=(\alph*)]
            \item Show that $R$ is an orthogonal matrix.
            \item Apply $R$ to the linear polarization state $|s\rangle = \begin{pmatrix}
                \cos\theta \\
                \sin\theta
            \end{pmatrix}$. Describe in everyday language the effect of this transformation on a state of linear polarization.
        \end{enumerate}
    \end{problem}
    \begin{answer}
        For part (a), we can determine whether $R$ is orthogonal determining whether $RR^T=I$: $$
            \begin{pmatrix}
                \cos\phi & -\sin\phi \\
                \sin\phi & \cos\phi
            \end{pmatrix}\begin{pmatrix}
                \cos\phi & \sin\phi \\
                -\sin\phi & \cos\phi
            \end{pmatrix} = \begin{pmatrix}
                \cos^2\phi + \sin^2\phi & \cos\phi\sin\phi - \sin\phi\cos\phi \\
                \cos\phi\sin\phi - \sin\phi\cos\phi & \cos^2\phi + \sin^2\phi
            \end{pmatrix}
        $$ and, via the Pythagorean identity, we can simplify the resulting matrix to the identity. Thus $R$ is an orthogonal matrix. Moving to part (b), we know that the resulting polarization state, $|s'\rangle$, will be $$
            R|s\rangle = \begin{pmatrix}
                \cos\phi & -\sin\phi \\
                \sin\phi & \cos\phi
            \end{pmatrix}\begin{pmatrix}
                \cos\theta \\
                \sin\theta
            \end{pmatrix} = \begin{pmatrix}
                \cos\phi\cos\theta - \sin\phi\sin\theta \\
                \sin\phi\cos\theta + \cos\phi\sin\theta
            \end{pmatrix} = \begin{pmatrix}
                \cos(\phi + \theta) \\
                \sin(\phi + \theta)
            \end{pmatrix}\footnotemark
        $$ which, evidently, is simply an increase in the angle of the linear polarization by $\phi$, resulting in a simple rotation.\footnotetext{I arrived at this final matrix using trigonometric identities provided in Problem 4 part (a).}
    \end{answer}
%\end{problemgroup}
\stepcounter{problem}
%\begin{problemgroup}
    \begin{problem}
        Let $R=\begin{pmatrix}
            \cos(\pi/(2n)) & -\sin(\pi/(2n)) \\
            \sin(\pi/(2n)) & \cos(\pi/(2n))
        \end{pmatrix}$.
        \begin{enumerate}[label=(\alph*)]
            \item Compute $R^n$. That is, compute the product of $n$ factors of $R$, where the multiplication is matrix multiplication. You may find the following trigonometric identities helpful: $\cos\alpha\cos\beta - \sin\alpha\sin\beta = \cos(\alpha + \beta)$; $\cos\alpha\sin\beta + \sin\alpha\cos\beta = \sin(\alpha + \beta)$.
            \item A horizontally polarized photon passes successively through $n$ small containers of sugar water, each of which effects the transformation $R$. The photon then encounters a polarizing filter whose preferred axis is horizontal. What is the probability of the photon passing the filter?
            \item Another horizontally polarized photon passes through the same $n$ containers of sugar water. But now, just after each container there is a polarizing filter whose preferred axis is horizontal. What is the probability that the photon will pass through all $n$ filters?
            \item Find the limit of your answer to part (c) as $n$ approaches infinity.
        \end{enumerate}
    \end{problem}
    \begin{answer}
        Starting with part (a), we can see from the Problem 2 part (b) that the standard rotation operation $R_\phi$, when applied to a photon's state, rotates it by $\phi$; this operation can be applied $n$ times to rotate the photon's state $n\phi$.\footnote{This applies to our newly assigned $R$ as the trigonometric structure is the same; i.e., the behavior is the same regardless of the provided arguments to the trigonometric functions as long as their positions and sign are the same.} In order to apply this operation $n$ times, we simply multiply the left hand side of the vector by $R^n$. Obfuscating computational steps provided in Problem 2 part (b), we can say, assuming some polarization state $|s\rangle = \left(\begin{smallmatrix}
            \cos\theta \\
            \sin\theta
        \end{smallmatrix}\right)$: $$
            R^n|s\rangle = \begin{pmatrix}
                \cos(\pi/(2n)) & -\sin(\pi/(2n)) \\
                \sin(\pi/(2n)) & \cos(\pi/(2n))
            \end{pmatrix}^n\begin{pmatrix}
                \cos\theta \\
                \sin\theta
            \end{pmatrix} = \begin{pmatrix}
                \cos(n\pi/(2n) + \theta) \\
                \sin(n\pi/(2n) + \theta)
            \end{pmatrix} = \begin{pmatrix}
                \cos(\nicefrac{\pi}{2} + \theta) \\
                \sin(\nicefrac{\pi}{2} + \theta)
            \end{pmatrix}
        $$
    \end{answer} demonstrating that our newly given $R^n$ will always rotate the polarization state by $\nicefrac{\pi}{2}$ radians if $n$ rotations are applied before inspection. Moving to part (b), we know that the photon's polarization state will be vertical after passing through $n$ containers. Because the horizontal polarizing filter is orthogonal to the photon's state, there is zero-chance that the photon will survive after a full rotation by the $n$ containers. For part (c), we first can establish that if a photon survives a container-filter combo according to a probability $p$, the photon's state polarization will be horizontal. This means that the total probability will be $p^n$ as the photon's chance of survival is equal at each consequtive filter. To determine $p^n$, we can say $$p = \langle s|m \rangle = |\cos(\pi/(2n)(1) + \sin(\pi/(2n)(0)|^2 = \cos^2(\pi/(2n)) $$$$p^n = \cos^{2n}(\pi/(2n)).$$ For part (d), we know that $\lim_{n\to \infty}\nicefrac{\pi}{2n}=0$ which implies that $\lim_{n\to \infty}\cos^{2n}(\pi/(2n))=1$. Intuitively, as the $n$-th slice gets smaller and smaller as $n$ increases, the photon's state is rotated less and less and a photon will consequently have a better chance to survive a polarized filter, increasing the overall chance of survival.
%\end{problemgroup}
\setcounter{subsubsection}{2}
\subsubsection{Circular and Elliptical Polarization}
\setcounter{problem}{0}
%\begin{problemgroup}
    \begin{problem}
        For each of the following state vectors, find a normalized vector that is orthogonal to the given vector. $$\begin{pmatrix}
            \nicefrac{\sqrt{3}}{2} \\
            \nicefrac{1}{2}
        \end{pmatrix}\quad\begin{pmatrix}
            \nicefrac{1}{\sqrt{2}} \\
            \nicefrac{i}{\sqrt{2}}
        \end{pmatrix}\quad\begin{pmatrix}
            \nicefrac{(1-i)}{2} \\
            \nicefrac{(1+i)}{2}
        \end{pmatrix}$$
    \end{problem}
    \begin{answer}
        We proceed with each vector by finding another vector in the same field such that $\langle s|t \rangle=0$:
        $$\begin{pmatrix}
            \nicefrac{\sqrt{3}}{2} \\
            \nicefrac{1}{2}
        \end{pmatrix}^\perp = \begin{pmatrix}
            -\nicefrac{1}{2} \\
            \nicefrac{\sqrt{3}}{2}
        \end{pmatrix};\quad\begin{pmatrix}
            \nicefrac{1}{\sqrt{2}} \\
            \nicefrac{i}{\sqrt{2}}
        \end{pmatrix}^\perp = \begin{pmatrix}
            \nicefrac{1}{\sqrt{2}} \\
            -\nicefrac{i}{\sqrt{2}}
        \end{pmatrix};\quad\begin{pmatrix}
            \nicefrac{(1-i)}{2} \\
            \nicefrac{(1+i)}{2}
        \end{pmatrix}^\perp = \begin{pmatrix}
            \nicefrac{(1+i)}{2} \\
            \nicefrac{(1-i)}{2}
        \end{pmatrix}$$
        Vectors in $\mathbb{R}$ are orthogonal if their dot-product is zero. However, for vectors in $\mathbb{C}$, we must determine if their inner product\footnote{This is called the \emph{Hermitian} inner product.}, $\langle u,v \rangle = \bar{u_1}v_1 + \bar{u_2}v_2$, is zero.
    \end{answer}
%\end{problemgroup}
\stepcounter{problem}
%\begin{problemgroup}
    \begin{problem}
        The rotation operation $R_\phi$, defined by $$R_\phi=\begin{pmatrix}
            \cos\phi & -\sin\phi \\
            \sin\phi & \cos\phi
        \end{pmatrix}$$ rotates any linear polarization state by an angle $\phi$. What does this transformation do to the right-hand circular polarization state? Is the resulting state a state of linear polarization, circular polarization, or elliptical polarization? Does the answer to this question depend on the value of $\phi$?
    \end{problem}
    \begin{answer}
        To start, let us define a standard right-hand circular polarization state as $|s\rangle=(\nicefrac{1}{\sqrt{2}})\left(\begin{smallmatrix}
            1 \\
            i
        \end{smallmatrix}\right)$.\footnote{We can ensure this is right-handed as $i=i(1)$ according to the quantum physics convention of using the receiver's point-of-view, which is also in-line with the textbook's definition.} We can then calculate the effect of the rotation operation on the photon's state:$$\frac{1}{\sqrt{2}}\begin{pmatrix}
            \cos\phi & -\sin\phi \\
            \sin\phi & \cos\phi
        \end{pmatrix}\begin{pmatrix}
            1 \\
            i
        \end{pmatrix} = \frac{1}{\sqrt{2}}\begin{pmatrix}
            \cos\phi - i\sin\phi \\
            \sin\phi + i\cos\phi
        \end{pmatrix} = \frac{1}{\sqrt{2}}\begin{pmatrix}
            e^{-i\phi} \\
            ie^{-i\phi}
        \end{pmatrix} = \left(\frac{e^{-i\phi}}{\sqrt{2}}\right)\begin{pmatrix}
            1 \\
            i
        \end{pmatrix}.$$ Because we are left with the same state vector multiplied by a complex factor, we can confirm that the state remains right-hand circular polarized for all $\phi$.
    \end{answer}
%\end{problemgroup}
%\begin{problemgroup}
\begin{problem}
    Consider the elliptical polarization represented by $|s\rangle = \begin{pmatrix}
        \nicefrac{1}{\sqrt{2}} \\
        \nicefrac{(1+i)}{2}
    \end{pmatrix}$. Suppose the measurement $M_\theta$ of Example 2.1.6 is applied to a photon in the state $|s\rangle$.
    \begin{enumerate}[label=(\alph*)]
        \item Find the probabilities of the two outcomes as functions of $\theta$.
        \item For what value of $\theta$ do the two probabilities differ the most from each other? The basis defined by $M_\theta$ for this value of $\theta$ can be thought of as giving the "principal axes" of the elliptical polarization.
    \end{enumerate}
\end{problem}
\begin{answer}
    Starting with part (a), let's restate the measurement: $$M_\theta=\left(\begin{pmatrix}
        \cos\theta \\
        \sin\theta
    \end{pmatrix},\begin{pmatrix}
        -\sin\theta \\
        \cos\theta
    \end{pmatrix}\right).$$ For probability $p_1$, we can say: $$p_1 = \langle s|m^{(1)} \rangle = \left|\frac{1}{\sqrt{2}}\cos\theta + \frac{1-i}{2}\sin\theta\right|^2.$$ We'll then need to derive its magnitude-squared expansion:\footnote{The general formula is $|a^2+b|^2=|a|^2+2\operatorname{Re}(a\bar{b})+|b|^2$ and is also sometimes referred to as the parallelogram law of complex numbers.} \begin{align*}
        p_1 &= \left|\frac{1}{\sqrt{2}}\cos\theta\right|^2 + 2\operatorname{Re}(\frac{1}{\sqrt{2}}\cos\theta\cdot \frac{1-i}{2}\sin\theta) + \left|\frac{1-i}{2}\sin\theta\right|^2 \\
        &= \frac{1}{2}\cos^2\theta + \frac{2\cos\theta\sin\theta}{2\sqrt{2}} + \frac{1}{2}\sin^2\theta \\
        &= \frac{1}{2} + \frac{\sin2\theta}{2\sqrt{2}}\ \therefore\  p_2 = \frac{1}{2} - \frac{\sin2\theta}{2\sqrt{2}}.
    \end{align*} For part (b), we can deduce by inspection that the difference in each probability will be maximized when $\frac{\sin2\theta}{2\sqrt{2}}$, which is when $\sin2\theta=\pm 1$. This occurs when $\theta = \pm\nicefrac{\pi}{4}$; thus the principal axes of the elliptical polarization exist at $\nicefrac{\pi}{4}$ and $\nicefrac{-\pi}{4}$.
\end{answer}
%\end{problemgroup}

\end{document}