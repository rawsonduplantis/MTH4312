% !TEX root = MTH4312HW2.tex
\documentclass[12pt]{article}
\usepackage[margin=1in]{geometry} 
\usepackage{amsmath,amsthm,amssymb,scrextend}
\usepackage{fancyhdr}
\pagestyle{fancy}

\newcommand{\cont}{\subseteq}
\usepackage{tikz}
\usepackage{pgfplots}
\usepackage{amsmath}
\usepackage[mathscr]{euscript}
\let\euscr\mathscr \let\mathscr\relax% just so we can load this and rsfs
\usepackage[scr]{rsfso}
\usepackage{amsthm}
\usepackage{caption}
\usepackage{amssymb}
\usepackage{enumitem}
\usepackage{multicol}
\usepackage{etoolbox}
\usepackage{tcolorbox}
\usepackage{mdframed}
\usepackage[table]{xcolor}
\usepackage[colorlinks=true, pdfstartview=FitV, linkcolor=blue,
citecolor=blue, urlcolor=blue]{hyperref}

\DeclareMathOperator{\arcsec}{arcsec}
\DeclareMathOperator{\arccot}{arccot}
\DeclareMathOperator{\arccsc}{arccsc}
\newcommand{\ddx}{\frac{d}{dx}}
\newcommand{\dfdx}{\frac{df}{dx}}
\newcommand{\ddxp}[1]{\frac{d}{dx}\left( #1 \right)}
\newcommand{\dydx}{\frac{dy}{dx}}
\let\ds\displaystyle
\newcommand{\intx}[1]{\int #1 \, dx}
\newcommand{\intt}[1]{\int #1 \, dt}
\newcommand{\defint}[3]{\int_{#1}^{#2} #3 \, dx}
\newcommand{\imp}{\Rightarrow}
\newcommand{\un}{\cup}
\newcommand{\inter}{\cap}
\newcommand{\ps}{\mathscr{P}}
\newcommand{\set}[1]{\left\{ #1 \right\}}

\newtheorem*{sol}{Solution}
\newtheorem*{claim}{Claim}
\newtheorem{problem}{Problem}
\theoremstyle{remark}  % Style for remarks
\newtheorem*{remark}{Remark}

\begin{document}
\setlength{\abovecaptionskip}{0pt} % Reduce space above caption
% EVERYTHING ABOVE THIS LINE IS JUST PREABLE, NO NEED TO MESS WITH IT.__________________________________________________________________________________________
%

\lhead{Rawson Duplantis}
\chead{MTH 4312: Cryptology}
\rhead{\today}

\section{Cryptography: An Overview}
\subsection{Elementary Ciphers}

%\begin{problemgroup}
    \begin{problem}
        We mentioned a substitution cipher in which each plaintext letter, represented by an integer $x$, is replaced by the letter corresponding to the integer $y=ax+b(\mod 26)$, where $a$ and $b$ are integers. If the alphabet we are using has $n$ letters, where $n$ is not necessarily 26, we can generalize this rule to $y=ax+b \pmod{n}$, where "$\mod n$" means that we take the remainder upon division by $n$. In answering the following questions, assume that the integers $a$ and $b$ are restricted to the values $0,\dots,n-1$.
        \begin{enumerate}[label=(\alph*)]
            \item Suppose that $n$ has the value 26, as it does if the plaintext is in English and we do not encrypt spaces or punctuation marks. Is there a reason not to use certain values of the constant $a$ or of the constant $b$? If so, which values are the bad ones and what makes them bad?
            \item If we also count "space" as a character to be encrypted, we have $n=27$. Now what, if any, are the bad values of $a$? Of $b$?
            \item For a general $n$, make a conjecture as to what will be the bad values of $a$ and $b$, if there are any.
        \end{enumerate}
    \end{problem}
    Starting with part (a), there are values of $a$ in particular that break the utility of substitution cipher. The values of $a$ that are \emph{bad} are those that are not relatively prime to 26, in this case $\{[2],[4],[6],[8],[10],[12],[14],[16],[18],[20],[22],[24]\}$. Moving to part (b), our procedure here doesn't change: find all values not relatively prime to 27. For 27, the \emph{bad} values are $\{[3], [6], [9], [12], [15], [18], [21], [24]\}$. Ending with the general case requested by part (c), we can say that unfit values for $a$ in the general substitution cipher formula $y=ax+b \pmod{n}$ are the zero divisors in the $\mathbb{Z}_n$ ring; conversely, the fit values are the units $U_n$. It's important to note that these two sets are disjoint and that $\mathbb{Z}_n\backslash\{0\}=U_n\mathbin{\dot{\bigcup}}\{\text{zero divisors of }\mathbb{Z}_n\}$ further reinforcing the \emph{bad} vs. \emph{good} paradigm in that a choice cannot be both unitary and a zero divisor. The units are uniquely \emph{good} choices for $a$ because they're invertible, ensuring a bijective (one-to-one and onto) mapping between the plaintext and the ciphertext which is crucial for decryption by Bob. Additionally, the choice $a=0$ while in $\mathbb{Z}_n$ is not a permitted choice in the first place as it maps every letter to the choice of $b$, destroying the plaintext for Bob. Finally, there are no restrictions on $b$ as any arbitrary linear shift won't change the bijectivity of the map between two sets of integers$\pmod{n}$.
%\end{problemgroup}
\vspace{5pt}
%\begin{problemgroup}
    \setcounter{problem}{2}
    \begin{problem}
    The following ciphertext was generated by a Vigenère cipher with a repeating key. All spaces and punctuation marks were removed from the plaintext, and the resulting ciphertext was broken into six-letter blocks.
    \begin{center}
        \begin{tabular}{lllllll}
            \texttt{NRUATW} & \texttt{YAHJSE} & \texttt{DIODII} & \texttt{TLWCIJ} & \texttt{DIOPRA} & \texttt{DPANTO} & \texttt{EOOPEG} \\
            \texttt{TNCWAS} & \texttt{DOBYAP} & \texttt{FRALLW} & \texttt{HSQNHW} & \texttt{DTDPIJ} & \texttt{GENDEO} & \texttt{BUWCEH} \\
            \texttt{LWKQGN} & \texttt{LVEEYZ} & \texttt{ZEOYOP} & \texttt{XAGPIP} & \texttt{DEHQOX} & \texttt{GIKFSE} & \texttt{YTDPOX} \\
            \texttt{DENGEZ} & \texttt{AHAYOI} & \texttt{PNWZNA} & \texttt{SAOEOH} & \texttt{ZOGQON} & \texttt{AAPEEN} & \texttt{YSWYDB} \\
            \texttt{TNZEHA} & \texttt{SIZOEJ} & \texttt{ZRZPRX} & \texttt{FTPSEN} & \texttt{PIOLNE} & \texttt{XPKCTW} & \texttt{YTZTFB} \\
            \texttt{PRAYCA} & \texttt{MEPHEA} & \texttt{YTDPSA} & \texttt{EWKAUN} & \texttt{DUEESE} & \texttt{YCNJPP} & \texttt{LNWWYO} \\
            \texttt{TSKYEG} & \texttt{YOSDTD} & \texttt{LTPSED} & \texttt{TDZPNK} & \texttt{CDACWW} & \texttt{DCKYSP} & \texttt{CUYEEZ} \\
            \texttt{MYDFMW} & \texttt{YIJEEH} & \texttt{WICPNY} & \texttt{PWDPRA} & \texttt{LSPSEK} & \texttt{CDACOB} & \texttt{YAPFRA} \\
            \texttt{LPLLRA} & \texttt{YTHJCK} & \texttt{XEOQRK} & \texttt{XAOZUN} & \texttt{NEKFTO} & \texttt{TDAZFK} & \texttt{FROPLR} \\
            \texttt{PSWYDE} & \texttt{DMKCEI} & \texttt{JSPPRE} & \texttt{ZUO} & \texttt{} & \texttt{} & \texttt{} \\
        \end{tabular}
    \end{center}
    \begin{enumerate}[label=(\alph*)]
        \item Look for strings of three or more letters that are repeated in the ciphertext. From the separations of different instances of the same string, try to infer the length of the key.
        \item \dots
    \end{enumerate}
    \end{problem}
    After briefly scanning the text, we'll use the trigram \textsc{TDP} as it appears three times in the 399-letter ciphertext.\footnote{Its important to note that there are five trigrams that appear three times: \texttt{PRA}, \texttt{RAL}, \texttt{DTD}, \texttt{PSE}, and our choice \texttt{TDP}. I quickly eliminated \texttt{DTD}, as the greatest common denominator between instances was 1 as no new information is revealed. While \texttt{PRA} and \texttt{RAL} have a $\gcd$ of 3 and 9, respectively, the six-letter partitioning presented by the textbook could be a hint that the $\gcd$ would also be 6, which was the case for \texttt{TDP} and \texttt{PSE}. This would also cover a key of period 3, meaning three of the five possible keys would be included in the solution. In the end, this process was \emph{not} brief.} It's important to remember that we aim to find the greatest common divisor of the distances between these instances, as we're predicting the key is cycling a set number of times between each instance.

    \begin{table}[h]
    \begin{center}
        \begin{tabular}{lllllll}
            \texttt{NRUATW} & \texttt{YAHJSE} & \texttt{DIODII} & \texttt{TLWCIJ} & \texttt{DIOPRA} & \texttt{DPANTO} & \texttt{EOOPEG} \\
            \texttt{TNCWAS} & \texttt{DOBYAP} & \texttt{FRALLW} & \texttt{HSQNHW} & \texttt{D\framebox[1\width]{TDP}IJ} & \texttt{GENDEO} & \texttt{BUWCEH} \\
            \texttt{LWKQGN} & \texttt{LVEEYZ} & \texttt{ZEOYOP} & \texttt{XAGPIP} & \texttt{DEHQOX} & \texttt{GIKFSE} & \texttt{Y\framebox[1\width]{TDP}OX} \\
            \texttt{DENGEZ} & \texttt{AHAYOI} & \texttt{PNWZNA} & \texttt{SAOEOH} & \texttt{ZOGQON} & \texttt{AAPEEN} & \texttt{YSWYDB} \\
            \texttt{TNZEHA} & \texttt{SIZOEJ} & \texttt{ZRZPRX} & \texttt{FTPSEN} & \texttt{PIOLNE} & \texttt{XPKCTW} & \texttt{YTZTFB} \\
            \texttt{PRAYCA} & \texttt{MEPHEA} & \texttt{Y\framebox[1\width]{TDP}SA} & \texttt{EWKAUN} & \texttt{DUEESE} & \texttt{YCNJPP} & \texttt{LNWWYO} \\
            \texttt{TSKYEG} & \texttt{YOSDTD} & \texttt{LTPSED} & \texttt{TDZPNK} & \texttt{CDACWW} & \texttt{DCKYSP} & \texttt{CUYEEZ} \\
            \texttt{MYDFMW} & \texttt{YIJEEH} & \texttt{WICPNY} & \texttt{PWDPRA} & \texttt{LSPSEK} & \texttt{CDACOB} & \texttt{YAPFRA} \\
            \texttt{LPLLRA} & \texttt{YTHJCK} & \texttt{XEOQRK} & \texttt{XAOZUN} & \texttt{NEKFTO} & \texttt{TDAZFK} & \texttt{FROPLR} \\
            \texttt{PSWYDE} & \texttt{DMKCEI} & \texttt{JSPPRE} & \texttt{ZUO} & \texttt{} & \texttt{} & \texttt{} \\
        \end{tabular}
    \end{center}
    \captionsetup{labelformat=empty}
    \caption*{The original ciphertext with all \texttt{TDP}s shown.}
    \end{table}

    The first appearance starts at the 67th letter, the second starts at 121st letter, and the final instance of the trigram appears at the 223rd letter. The distance between the first two instances is 54, and the distance between the last two instances is 102. The factors of 54 are $\{1, 2, 3, 6, 9, 18, 27, 54\}$ while the factors of 102 are $\{1, 2, 3, 6, 17, 34, 51, 102\}$. Consequently, $\gcd(54, 102)=6$, so \underline{the periodic key is potentially 6 characters long.}
%\end{problemgroup}

\subsection{Enigma}

%\begin{problemgroup}
    \begin{problem}
            This is a counting problem focusing on the Enigma plugboard. Recall that the plugboard permutation interchanges some of the letters in pairs. For example, A and F might be interchanged, and M and X might be interchanged.
            \begin{enumerate}[label=(\alph*)]
                \item Suppose that only one pair of letters are interchanged and the other 24 letters are left unchanged. How many ways are there of choosing the special pair?
                \item In the standard Enigma machine, six pairs of letters were swapped. How many ways are there of choosing these six pairs? Does your answer agree with our rough estimate of $10^{11}$?
            \end{enumerate}
    \end{problem}
    Starting with (a), in order to select 2 plugs from a 26 plug plugboard without regard to their order, we can express this as $${26\choose2} =\frac{26!}{2!\times24!}=325\text{ ways}$$ to choose special pairs. For (b), we increase the number of pairs to six, meaning we're now picking 12 plugs instead of 2 resulting in ${26\choose12}$. However, we also need to eliminate the duplicate instances of pairs with reverse order \emph{and} eliminate the duplicate instances of pairs organized in a different order, both of which aren't relevant to our question. This is then $${26\choose12} = \frac{26!}{12!\times 12!} \times \frac{12!}{6!\times(2!)^6}\approx5.5\times10^{11}\text{ ways}$$ to choose special pairs, confirming the given estimate of $10^{11}$ choices.
%\end{problemgroup}

\setcounter{subsection}{8}
\subsection{RSA}

%\begin{problemgroup}
    \setcounter{problem}{2}
    \begin{problem}
        Prove that $r_{k+1}$ in the Euclidean Algorithm is the greatest common divisor of $a$ and $b$.
    \end{problem}
    \begin{proof}
        Suppose $a,b\in\mathbb{Z}$ such that $a\geq 0$, $b>0$. We'll show using the Euclidean Algorithm that its penultimate remainder is a common divisor of $a$ and $b$ and that all other common divisors divide this penultimate remainder. Starting with the expression from the theorem of the Division Algorithm:
        \begin{align*}
            a=q_1b+r_1
        \end{align*}
    \end{proof}

%\end{problemgroup}

\iffalse
% \maketitle
\section{Intro to LaTeX}
Hi there! Here is a brief introduction to LaTeX. It's a great system for creating beautifully typeset scientific documents. The main difference between LaTeX and your standard word processing program, is that you write code that tells LaTeX what you want it to display and the program creates handles all the little things that make your math look great.

Lets start with some basics. You can type text normally but when you want to type some math start and end your expression with a \$. For example if I type the Pythagorean Theorem, as long as I surround my equation with dollar signs I get $a^2 +b^2 = c^2$. If I want to emphasize an equation simply put TWO dollar signs around the mathematics. For example a similar expression with two dollar signs around it looks like so: $$x_1^n + x_2^n = x_3^n.$$ (check out the code to see how to generate subscripts!)

In LaTeX we use the curly braces, \{ \}, to group things. If we want to raise $e$ to the $42$nd power and we just type it with no grouping we get $e^42$ which is not what we want. Surrounding the $42$ with curly braces gives $e^{42}$.

The really really really great thing about latex is the HUGE library of mathematical symbols, every symbol starts with a $\backslash$, so if I want a nice pi, I just type $\backslash$pi in math mode like so: $\pi$. Want an integral? Try $\backslash$int, want a fraction? Try $\backslash$frac\{a\}\{b\} (this gives $\frac{a}{b}$) Here is a more complicated example:
$$
\int_a^b f \left(\frac{x}{2} \right) \ dx =2\left( F\left(\frac{b}{2} \right) - F\left(\frac{a}{2}\right) \right)
$$
(those nice sized parenthesis come from using the $\backslash$left( and $\backslash$right) commands). If you are wondering what the latex command for a symbol is, just try and guess it. If that doesn't work, google it!

There are a few little tricks to making your math look nice, one is the align environment which lets you type multiple lines of aligned mathematical expressions. For instance:
\begin{align*}
(x^2+y^2)(z^2+w^2) & = (xz)^2 + (yz)^2 + (xw)^2 + (yw)^2 \\
& =  (xz)^2 + 2wxyz + (yw)^2 + (yz)^2 - 2 wxyz + (xw)^2 \\
&= (xz+yw)^2 + (yz-xw)^2
\end{align*}
here the $\backslash\backslash$ is a line break and the $\&$ is an alignment character. 
\section{Sample Proofs}
\begin{problem} $1^3+2^3+\ldots +n^3 = \left[ \frac{n(n+1)}{2}\right]^2$  for all natural numbers. 

\end{problem}
 
\begin{proof}
We proceed by induction. When $n=1$ we have 
$$
1^3 + \ldots + n^3 = 1^3 =1
$$
and
$$
\left[ \frac{n(n+1)}{2}\right]^2=\left[ \frac{1(1+1)}{2}\right]^2 = 1^2 =1.
$$
Thus the equation holds when $n=1$.

Now assume the equation holds for some $n \in \mathbb N$. Then by our assumption we have
\begin{align*}
1^3 + \ldots + n^3 + (n+1)^3& =   \left[ \frac{n(n+1)}{2}\right]^2 + (n+1)^3 = (n+1)^2\left( \frac{n^2}{4} + (n+1)\right) \\
&=(n+1)^2\left( \frac{n^2+4n+4}{4} \right) = (n+1)^2\frac{(n+2)^2}{4}  \\ &=  \left[ \frac{(n+1)(n+2)}{2}\right]^2.
\end{align*}
Therefore, by induction, the equation holds for all $n\in \mathbb N$.
\end{proof}

\begin{problem} Prove $A \subseteq B \Rightarrow A \cap C \cont B \cap C$
\end{problem}
\begin{proof} Assume $A \cont B $ and let $x\in A \cap C$. If $x \in  A \cap C$, then $x\in A$ and $x \in C$. As $x\in A$, by assumption we have that $x \in B$. Thus $x \in B$ and $x \in C$, giving $x\in B \cap C$. Therefore if $A \subseteq B $, then $A \cap C \cont B \cap C$.
\end{proof}

% THE DOCUMENT IS ESSENTIALLY DONE AT THIS POINT, NO NEED TO EDIT ANYTHING BELOW THIS______________________________________________________________________________________________
 
\fi
\end{document}