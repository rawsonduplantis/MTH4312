% !TEX root = MTH4312HW1.tex
\documentclass[12pt]{article}
\usepackage[margin=1in]{geometry} 
\usepackage{amsmath,amsthm,amssymb,scrextend}
\usepackage{fancyhdr}
\pagestyle{fancy}

\newcommand{\cont}{\subseteq}
\usepackage{tikz}
\usepackage{pgfplots}
\usepackage{amsmath}
\usepackage[mathscr]{euscript}
\let\euscr\mathscr \let\mathscr\relax% just so we can load this and rsfs
\usepackage[scr]{rsfso}
\usepackage{amsthm}
\usepackage{caption}
\usepackage{amssymb}
\usepackage{enumitem}
\usepackage{multicol}
\usepackage{etoolbox}
\usepackage{tcolorbox}
\usepackage{mdframed}
\usepackage[table]{xcolor}
\usepackage[colorlinks=true, pdfstartview=FitV, linkcolor=blue,
citecolor=blue, urlcolor=blue]{hyperref}

\DeclareMathOperator{\arcsec}{arcsec}
\DeclareMathOperator{\arccot}{arccot}
\DeclareMathOperator{\arccsc}{arccsc}
\newcommand{\ddx}{\frac{d}{dx}}
\newcommand{\dfdx}{\frac{df}{dx}}
\newcommand{\ddxp}[1]{\frac{d}{dx}\left( #1 \right)}
\newcommand{\dydx}{\frac{dy}{dx}}
\let\ds\displaystyle
\newcommand{\intx}[1]{\int #1 \, dx}
\newcommand{\intt}[1]{\int #1 \, dt}
\newcommand{\defint}[3]{\int_{#1}^{#2} #3 \, dx}
\newcommand{\imp}{\Rightarrow}
\newcommand{\un}{\cup}
\newcommand{\inter}{\cap}
\newcommand{\ps}{\mathscr{P}}
\newcommand{\set}[1]{\left\{ #1 \right\}}

\newtheorem*{sol}{Solution}
\newtheorem*{claim}{Claim}
\newtheorem{problem}{Problem}
\theoremstyle{remark}  % Style for remarks
\newtheorem*{remark}{Remark}

\begin{document}
\setlength{\abovecaptionskip}{0pt} % Reduce space above caption
% EVERYTHING ABOVE THIS LINE IS JUST PREABLE, NO NEED TO MESS WITH IT.__________________________________________________________________________________________
%

\lhead{Rawson Duplantis}
\chead{MTH 4312: Cryptology}
\rhead{\today}

\section{1.1: Elementary Ciphers}
%\begin{problemgroup}
    \stepcounter{problem}
    \begin{problem}
        The following ciphertext was generated by a Vigenère cipher with a repeating key. All spaces and punctuation marks were removed from the plaintext, and the resulting ciphertext was broken into six-letter blocks.
        \begin{center}
            \begin{tabular}{lllllll}
                \texttt{NRUATW} & \texttt{YAHJSE} & \texttt{DIODII} & \texttt{TLWCIJ} & \texttt{DIOPRA} & \texttt{DPANTO} & \texttt{EOOPEG} \\
                \texttt{TNCWAS} & \texttt{DOBYAP} & \texttt{FRALLW} & \texttt{HSQNHW} & \texttt{DTDPIJ} & \texttt{GENDEO} & \texttt{BUWCEH} \\
                \texttt{LWKQGN} & \texttt{LVEEYZ} & \texttt{ZEOYOP} & \texttt{XAGPIP} & \texttt{DEHQOX} & \texttt{GIKFSE} & \texttt{YTDPOX} \\
                \texttt{DENGEZ} & \texttt{AHAYOI} & \texttt{PNWZNA} & \texttt{SAOEOH} & \texttt{ZOGQON} & \texttt{AAPEEN} & \texttt{YSWYDB} \\
                \texttt{TNZEHA} & \texttt{SIZOEJ} & \texttt{ZRZPRX} & \texttt{FTPSEN} & \texttt{PIOLNE} & \texttt{XPKCTW} & \texttt{YTZTFB} \\
                \texttt{PRAYCA} & \texttt{MEPHEA} & \texttt{YTDPSA} & \texttt{EWKAUN} & \texttt{DUEESE} & \texttt{YCNJPP} & \texttt{LNWWYO} \\
                \texttt{TSKYEG} & \texttt{YOSDTD} & \texttt{LTPSED} & \texttt{TDZPNK} & \texttt{CDACWW} & \texttt{DCKYSP} & \texttt{CUYEEZ} \\
                \texttt{MYDFMW} & \texttt{YIJEEH} & \texttt{WICPNY} & \texttt{PWDPRA} & \texttt{LSPSEK} & \texttt{CDACOB} & \texttt{YAPFRA} \\
                \texttt{LPLLRA} & \texttt{YTHJCK} & \texttt{XEOQRK} & \texttt{XAOZUN} & \texttt{NEKFTO} & \texttt{TDAZFK} & \texttt{FROPLR} \\
                \texttt{PSWYDE} & \texttt{DMKCEI} & \texttt{JSPPRE} & \texttt{ZUO} & \texttt{} & \texttt{} & \texttt{} \\
            \end{tabular}
        \end{center}
        \begin{enumerate}[label=(\alph*)]
            \item Look for strings of three or more letters that are repeated in the ciphertext. From the separations of different instances of the same string, try to infer the length of the key.
            \item Using frequency analysis or any other means, try to find the key and the plaintext.
        \end{enumerate}
    \end{problem}
    After briefly scanning the text, we'll use the trigram \textsc{TDP} as it appears three times in the 399-letter ciphertext.\footnote{Its important to note that there are five trigrams that appear three times: \texttt{PRA}, \texttt{RAL}, \texttt{DTD}, \texttt{PSE}, and our choice \texttt{TDP}. I quickly eliminated \texttt{DTD}, as the greatest common denominator between instances was 1. While \texttt{PRA} and \texttt{RAL} have a $\gcd$ of 3 and 9, respectively, the six-letter partitioning presented by the textbook could be a hint that the $\gcd$ would also be 6, which was the case for \texttt{TDP} and \texttt{PSE}. This would also cover a key of period 3, meaning three of the five possible keys would be included in the solution. In the end, this process was \emph{not} brief.}

    \begin{table}[h]
        \begin{center}
            \begin{tabular}{lllllll}
                \texttt{NRUATW} & \texttt{YAHJSE} & \texttt{DIODII} & \texttt{TLWCIJ} & \texttt{DIOPRA} & \texttt{DPANTO} & \texttt{EOOPEG} \\
                \texttt{TNCWAS} & \texttt{DOBYAP} & \texttt{FRALLW} & \texttt{HSQNHW} & \texttt{D\framebox[1\width]{TDP}IJ} & \texttt{GENDEO} & \texttt{BUWCEH} \\
                \texttt{LWKQGN} & \texttt{LVEEYZ} & \texttt{ZEOYOP} & \texttt{XAGPIP} & \texttt{DEHQOX} & \texttt{GIKFSE} & \texttt{Y\framebox[1\width]{TDP}OX} \\
                \texttt{DENGEZ} & \texttt{AHAYOI} & \texttt{PNWZNA} & \texttt{SAOEOH} & \texttt{ZOGQON} & \texttt{AAPEEN} & \texttt{YSWYDB} \\
                \texttt{TNZEHA} & \texttt{SIZOEJ} & \texttt{ZRZPRX} & \texttt{FTPSEN} & \texttt{PIOLNE} & \texttt{XPKCTW} & \texttt{YTZTFB} \\
                \texttt{PRAYCA} & \texttt{MEPHEA} & \texttt{Y\framebox[1\width]{TDP}SA} & \texttt{EWKAUN} & \texttt{DUEESE} & \texttt{YCNJPP} & \texttt{LNWWYO} \\
                \texttt{TSKYEG} & \texttt{YOSDTD} & \texttt{LTPSED} & \texttt{TDZPNK} & \texttt{CDACWW} & \texttt{DCKYSP} & \texttt{CUYEEZ} \\
                \texttt{MYDFMW} & \texttt{YIJEEH} & \texttt{WICPNY} & \texttt{PWDPRA} & \texttt{LSPSEK} & \texttt{CDACOB} & \texttt{YAPFRA} \\
                \texttt{LPLLRA} & \texttt{YTHJCK} & \texttt{XEOQRK} & \texttt{XAOZUN} & \texttt{NEKFTO} & \texttt{TDAZFK} & \texttt{FROPLR} \\
                \texttt{PSWYDE} & \texttt{DMKCEI} & \texttt{JSPPRE} & \texttt{ZUO} & \texttt{} & \texttt{} & \texttt{} \\
            \end{tabular}
        \end{center}
        \captionsetup{labelformat=empty}
        \caption*{The original ciphertext with all \texttt{TDP}s shown.}
    \end{table}

    The first appearance starts at the 67th letter, the second starts at 121st letter, and the final instance of the trigram appears at the 223rd letter. The distance between the first two instances is 54, and the distance between the last two instances is 102. The factors of 54 are $\{1, 2, 3, 6, 9, 18, 27, 54\}$ while the factors of 102 are $\{1, 2, 3, 6, 17, 34, 51, 102\}$. Consequently, $\gcd(54, 102)=6$, so the periodic key is potentially 6 characters long. We'll take every sixth letter of the ciphertext and count their frequency.\footnote{This was done using Python; script found \hyperlink{https://github.com/rawsonduplantis/MTH4312/blob/main/HW1/hw1.py}{here}.} \par The first column of frequency-distribution pairs represents the first character in the hexagram, the second column the second letter, and so on.
    \begin{table}[h]
        \begin{center}
            \begin{tabular}{c|cc|cc|cc|cc|cc|cc|}
                Letter & \# 1 & \% 1 & \# 2 & \% 2 & \# 3 & \% 3 & \# 4 & \% 4 & \# 5 & \% 5 & \# 6 & \% 6 \\
                \hline
                A & 2 & 3.0 & 6 & 9.0 & 7 & 10.4 & 2 & 3.0 & 2 & 3.0 & \textbf{9} & \textbf{13.6} \\
                B & 1 & 1.5 & 0 & 0.0 & 1 & 1.5 & 0 & 0.0 & 0 & 0.0 & 3 & 4.5 \\
                C & 3 & 4.5 & 2 & 3.0 & 2 & 3.0 & 6 & 9.1 & 2 & 3.0 & 0 & 0.0 \\
                D & \textbf{10} & \textbf{14.9} & 4 & 6.0 & 5 & 7.5 & 3 & 4.5 & 2 & 3.0 & 2 & 3.0 \\
                E & 2 & 3.0 & \textbf{7} & \textbf{10.4} & 2 & 3.0 & 7 & 10.6 & \textbf{14} & \textbf{21.2} & 6 & 9.1 \\
                F & 3 & 4.5 & 0 & 0.0 & 0 & 0.0 & 4 & 6.1 & 2 & 3.0 & 0 & 0.0 \\
                G & 2 & 3.0 & 0 & 0.0 & 2 & 3.0 & 1 & 1.5 & 1 & 1.5 & 2 & 3.0 \\
                H & 1 & 1.5 & 1 & 1.5 & 3 & 4.5 & 1 & 1.5 & 2 & 3.0 & 3 & 4.5 \\
                I & 0 & 0.0 & \textbf{7} & \textbf{10.4} & 0 & 0.0 & 0 & 0.0 & 4 & 6.1 & 3 & 4.5 \\
                J & 1 & 1.5 & 0 & 0.0 & 1 & 1.5 & 3 & 4.5 & 0 & 0.0 & 3 & 4.5 \\
                K & 0 & 0.0 & 0 & 0.0 & \textbf{8} & \textbf{11.9} & 0 & 0.0 & 0 & 0.0 & 5 & 7.6 \\
                L & 6 & 9.0 & 1 & 1.5 & 1 & 1.5 & 3 & 4.5 & 2 & 3.0 & 0 & 0.0 \\
                M & 2 & 3.0 & 1 & 1.5 & 0 & 0.0 & 0 & 0.0 & 1 & 1.5 & 0 & 0.0 \\
                N & 2 & 3.0 & 4 & 6.0 & 3 & 4.5 & 2 & 3.0 & 4 & 6.1 & 6 & 9.1 \\
                O & 0 & 0.0 & 4 & 6.0 & \textbf{10} & \textbf{14.9} & 1 & 1.5 & 7 & 10.6 & 4 & 6.1 \\
                P & 5 & 7.5 & 3 & 4.5 & 7 & 10.4 & \textbf{12} & \textbf{18.2} & 1 & 1.5 & 5 & 7.6 \\
                Q & 0 & 0.0 & 0 & 0.0 & 1 & 1.5 & 4 & 6.1 & 0 & 0.0 & 0 & 0.0 \\
                R & 0 & 0.0 & 5 & 7.5 & 0 & 0.0 & 0 & 0.0 & 7 & 10.6 & 1 & 1.5 \\
                S & 2 & 3.0 & 6 & 9.0 & 1 & 1.5 & 3 & 4.5 & 5 & 7.6 & 1 & 1.5 \\
                T & 6 & 9.0 & \textbf{7} & \textbf{10.4} & 0 & 0.0 & 1 & 1.5 & 5 & 7.6 & 0 & 0.0 \\
                U & 0 & 0.0 & 4 & 6.0 & 1 & 1.5 & 0 & 0.0 & 2 & 3.0 & 0 & 0.0 \\
                V & 0 & 0.0 & 1 & 1.5 & 0 & 0.0 & 0 & 0.0 & 0 & 0.0 & 0 & 0.0 \\
                W & 1 & 1.5 & 3 & 4.5 & 6 & 9.0 & 2 & 3.0 & 1 & 1.5 & 6 & 9.1 \\
                X & 4 & 6.0 & 0 & 0.0 & 0 & 0.0 & 0 & 0.0 & 0 & 0.0 & 3 & 4.5 \\
                Y & \textbf{10} & \textbf{14.9} & 1 & 1.5 & 1 & 1.5 & \textbf{8} & \textbf{12.1} & 2 & 3.0 & 1 & 1.5 \\
                Z & 4 & 6.0 & 0 & 0.0 & 5 & 7.5 & 3 & 4.5 & 0 & 0.0 & 3 & 4.5 \\
            \end{tabular}
        \end{center}
        \captionsetup{labelformat=empty}
        \caption*{Counts and frequencies of ciphertext character by their hexagram position.}
    \end{table}
    We start with the original trigram, \texttt{TDP}. Since our process depends on this trigram representing a common trigram in English, we should aim for one when we attempt difference shifts in each position of the hexagram in addition to using frequency analysis as an aid. The frequency-distribution pair table above has the most common characters bolded. \par Starting with the most common English trigram, \texttt{THE}, we can see not only that our common cipher trigram starts with a \texttt{T} as well, but that it also could feasibly not be a shifted position via the reasonable frequency of \texttt{E}, \texttt{I}, and \texttt{T}. The procedure continues with guessing a shift of 4 in the third position, as that would correspond to a mapping \texttt{D} $\mapsto$ \texttt{H}, our second target letter. Finally, after seeing that our ciphertext's \texttt{P} is the most common letter in the fourth position, we can reasonably guess \texttt{TDP} $\mapsto$ \texttt{THE}. We obtain the following partially deciphered text.
    \begin{table}[h]
        \begin{center}
            \begin{tabular}{lllllll}
                \texttt{NRYPTW} & \texttt{YALYSE} & \texttt{DISSII} & \texttt{TLARIJ} & \texttt{DISERA} & \texttt{DPECTO} & \texttt{EOSEEG} \\
                \texttt{TNGLAS} & \texttt{DOFNAP} & \texttt{FREALW} & \texttt{HSUCHW} & \texttt{DTHEIJ} & \texttt{GERSEO} & \texttt{BUAREH} \\
                \texttt{LWOFGN} & \texttt{LVITYZ} & \texttt{ZESNOP} & \texttt{XAKEIP} & \texttt{DELFOX} & \texttt{GIOUSE} & \texttt{YTHEOX} \\
                \texttt{DERVEZ} & \texttt{AHENOI} & \texttt{PNAONA} & \texttt{SASTOH} & \texttt{ZOKFON} & \texttt{AATTEN} & \texttt{YSANDB} \\
                \texttt{TNDTHA} & \texttt{SIDDEJ} & \texttt{ZRDERX} & \texttt{FTTHEN} & \texttt{PISANE} & \texttt{XPORTW} & \texttt{YTDIFB} \\
                \texttt{PRENCA} & \texttt{METWEA} & \texttt{YTHESA} & \texttt{EWOPUN} & \texttt{DUITSE} & \texttt{YCRYPP} & \texttt{LNALYO} \\
                \texttt{TSONEG} & \texttt{YOWSTD} & \texttt{LTTHED} & \texttt{TDDENK} & \texttt{CDERWW} & \texttt{DCONSP} & \texttt{CUCTEZ} \\
                \texttt{MYHUMW} & \texttt{YINTEH} & \texttt{WIGENY} & \texttt{PWHERA} & \texttt{LSTHEK} & \texttt{CDEROB} & \texttt{YATURA} \\
                \texttt{LPPARA} & \texttt{YTLYCK} & \texttt{XESFRK} & \texttt{XASOUN} & \texttt{NEOUTO} & \texttt{TDEOFK} & \texttt{FRSELR} \\
                \texttt{PSANDE} & \texttt{DMOREI} & \texttt{JSTERE} & \texttt{ZUS} & & & \\
            \end{tabular}
        \end{center}
        \captionsetup{labelformat=empty}
        \caption*{Partially deciphered ciphertext using shifts of $\{0,\mathbf{0},\mathbf{4},\mathbf{-11},0,0\}$}
    \end{table}
    After verifying by inspection that the shifted characters could lead to a valid English message, we complete our procedure with frequency analysis. Positions five and six each have a letter significantly more frequent than the rest, which we'll assume is \texttt{E} in our plaintext. Finally, we can complete our deciphering attempt by checking if we can choose a reasonably frequent letter to complete our first word. In this case, we have \texttt{NRYPTA}\dots which most likely is \texttt{CRYPTA}\dots , especially taking context into consideration. We have our final set of deciphered hexagrams!
    \begin{table}[h]
        \begin{center}
            \begin{tabular}{lllllll}
                \texttt{CRYPTA} & \texttt{NALYSI} & \texttt{SISSIM} & \texttt{ILARIN} & \texttt{SISERE} & \texttt{SPECTS} & \texttt{TOSEEK} \\
                \texttt{INGLAW} & \texttt{SOFNAT} & \texttt{UREALA} & \texttt{WSUCHA} & \texttt{STHEIN} & \texttt{VERSES} & \texttt{QUAREL} \\
                \texttt{AWOFGR} & \texttt{AVITYD} & \texttt{OESNOT} & \texttt{MAKEIT} & \texttt{SELFOB} & \texttt{VIOUSI} & \texttt{NTHEOB} \\
                \texttt{SERVED} & \texttt{PHENOM} & \texttt{ENAONE} & \texttt{HASTOL} & \texttt{OOKFOR} & \texttt{PATTER} & \texttt{NSANDF} \\
                \texttt{INDTHE} & \texttt{HIDDEN} & \texttt{ORDERB} & \texttt{UTTHER} & \texttt{EISANI} & \texttt{MPORTA} & \texttt{NTDIFF} \\
                \texttt{ERENCE} & \texttt{BETWEE} & \texttt{NTHESE} & \texttt{TWOPUR} & \texttt{SUITSI} & \texttt{NCRYPT} & \texttt{ANALYS} \\
                \texttt{ISONEK} & \texttt{NOWSTH} & \texttt{ATTHEH} & \texttt{IDDENO} & \texttt{RDERWA} & \texttt{SCONST} & \texttt{RUCTED} \\
                \texttt{BYHUMA} & \texttt{NINTEL} & \texttt{LIGENC} & \texttt{EWHERE} & \texttt{ASTHEO} & \texttt{RDEROF} & \texttt{NATURE} \\
                \texttt{APPARE} & \texttt{NTLYCO} & \texttt{MESFRO} & \texttt{MASOUR} & \texttt{CEOUTS} & \texttt{IDEOFO} & \texttt{URSELV} \\
                \texttt{ESANDI} & \texttt{SMOREM} & \texttt{YSTERI} & \texttt{OUS} & & & \\
            \end{tabular}
        \end{center}
        \captionsetup{labelformat=empty}
        \caption*{Our deciphered hexagrams!}
    \end{table}

    The key, $\{15, 0, 4\}$, corresponds to \texttt{PAE} and is notably only 3 letters long.
    \begin{quote}
        Cryptanalysis is similar in respects to seeking laws of nature. A law such as the inverse square law of gravity does not make itself obvious in the observed phenomena. One has to look for patterns and find the hidden order. But there is an important difference between these two pursuits. In cryptanalysis, one knows that the hidden order was constructed by human intelligence,whereas the order of nature apparently comes from a source outside of ourselves and is more mysterious.
    \end{quote}

%\end{problemgroup}

\iffalse
% \maketitle
\section{Intro to LaTeX}
Hi there! Here is a brief introduction to LaTeX. It's a great system for creating beautifully typeset scientific documents. The main difference between LaTeX and your standard word processing program, is that you write code that tells LaTeX what you want it to display and the program creates handles all the little things that make your math look great.

Lets start with some basics. You can type text normally but when you want to type some math start and end your expression with a \$. For example if I type the Pythagorean Theorem, as long as I surround my equation with dollar signs I get $a^2 +b^2 = c^2$. If I want to emphasize an equation simply put TWO dollar signs around the mathematics. For example a similar expression with two dollar signs around it looks like so: $$x_1^n + x_2^n = x_3^n.$$ (check out the code to see how to generate subscripts!)

In LaTeX we use the curly braces, \{ \}, to group things. If we want to raise $e$ to the $42$nd power and we just type it with no grouping we get $e^42$ which is not what we want. Surrounding the $42$ with curly braces gives $e^{42}$.

The really really really great thing about latex is the HUGE library of mathematical symbols, every symbol starts with a $\backslash$, so if I want a nice pi, I just type $\backslash$pi in math mode like so: $\pi$. Want an integral? Try $\backslash$int, want a fraction? Try $\backslash$frac\{a\}\{b\} (this gives $\frac{a}{b}$) Here is a more complicated example:
$$
\int_a^b f \left(\frac{x}{2} \right) \ dx =2\left( F\left(\frac{b}{2} \right) - F\left(\frac{a}{2}\right) \right)
$$
(those nice sized parenthesis come from using the $\backslash$left( and $\backslash$right) commands). If you are wondering what the latex command for a symbol is, just try and guess it. If that doesn't work, google it!

There are a few little tricks to making your math look nice, one is the align environment which lets you type multiple lines of aligned mathematical expressions. For instance:
\begin{align*}
(x^2+y^2)(z^2+w^2) & = (xz)^2 + (yz)^2 + (xw)^2 + (yw)^2 \\
& =  (xz)^2 + 2wxyz + (yw)^2 + (yz)^2 - 2 wxyz + (xw)^2 \\
&= (xz+yw)^2 + (yz-xw)^2
\end{align*}
here the $\backslash\backslash$ is a line break and the $\&$ is an alignment character. 
\section{Sample Proofs}
\begin{problem} $1^3+2^3+\ldots +n^3 = \left[ \frac{n(n+1)}{2}\right]^2$  for all natural numbers. 

\end{problem}
 
\begin{proof}
We proceed by induction. When $n=1$ we have 
$$
1^3 + \ldots + n^3 = 1^3 =1
$$
and
$$
\left[ \frac{n(n+1)}{2}\right]^2=\left[ \frac{1(1+1)}{2}\right]^2 = 1^2 =1.
$$
Thus the equation holds when $n=1$.

Now assume the equation holds for some $n \in \mathbb N$. Then by our assumption we have
\begin{align*}
1^3 + \ldots + n^3 + (n+1)^3& =   \left[ \frac{n(n+1)}{2}\right]^2 + (n+1)^3 = (n+1)^2\left( \frac{n^2}{4} + (n+1)\right) \\
&=(n+1)^2\left( \frac{n^2+4n+4}{4} \right) = (n+1)^2\frac{(n+2)^2}{4}  \\ &=  \left[ \frac{(n+1)(n+2)}{2}\right]^2.
\end{align*}
Therefore, by induction, the equation holds for all $n\in \mathbb N$.
\end{proof}

\begin{problem} Prove $A \subseteq B \Rightarrow A \cap C \cont B \cap C$
\end{problem}
\begin{proof} Assume $A \cont B $ and let $x\in A \cap C$. If $x \in  A \cap C$, then $x\in A$ and $x \in C$. As $x\in A$, by assumption we have that $x \in B$. Thus $x \in B$ and $x \in C$, giving $x\in B \cap C$. Therefore if $A \subseteq B $, then $A \cap C \cont B \cap C$.
\end{proof}

% THE DOCUMENT IS ESSENTIALLY DONE AT THIS POINT, NO NEED TO EDIT ANYTHING BELOW THIS______________________________________________________________________________________________
 
\fi
\end{document}